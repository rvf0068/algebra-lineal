\documentclass[12pt]{exam}

\usepackage[spanish,mexico]{babel}
\usepackage[latin1]{inputenc}
\usepackage[T1]{fontenc}

\usepackage{fullpage}
\usepackage{amsmath,amsfonts}
\usepackage{enumerate}
\pagestyle{head}
\extraheadheight{.8in}

\lhead{Algebra Lineal, LIFTA, UAEH.\\
Primer Examen,\\
2 de septiembre de 2013}

\usepackage{fourier}

\newcommand{\RR}{\mathbb{R}}
\newcommand{\setof}[2]{\left\{\,#1\mid #2\,\right\}}

\begin{document}
\hbox to \textwidth{NOMBRE:\enspace\hrulefill}
\bigskip{}

Recuerda que debes explicar tus respuestas. Respuestas con poca o nula
argumentaci�n adecuada recibir�n poco o nada de cr�dito. Escoge 5
preguntas, marcando de manera clara las preguntas
seleccionadas. Tienes dos horas para resolverlo.

\begin{questions}
  \question Determina si el conjunto
  \begin{equation*}
    \setof{(x,y,z)\in\RR^{3}}{x=z}
  \end{equation*}
  es un subespacio de $\RR^3$.

  \question Determina si el conjunto
  \begin{equation*}
    \setof{(x,y,z)\in\RR^{3}}{2x-3y-z=x+4y-z}
  \end{equation*}
  es un subespacio de $\RR^3$.

  \question Resuelve el sistema de ecuaciones:
  \begin{alignat*}{3}
    x &+2y &-z &=2\\
    2x&+4y &+z &=7\\
    3x&+6y &-2z&=7\\
  \end{alignat*}

  \question �Verdadero o falso? Sea $V$ un espacio vectorial sobre un
  campo $F$. Entonces, si $W_{1}$ y $W_{2}$ son subespacios de $V$, se
  tiene que $W_{1}\cap W_{2}$ es subespacio de $V$.

  \question Determina si el vector $(1,2,3)$ es combinaci�n lineal del
  conjunto de vectores $\{(4,5,6),(7,8,9)\}$.

  \question �Verdadero o falso? En el espacio vectorial $\RR[x]$, el
  vector $x^{2}+1$ est� en el subespacio generado por
  $\{x^{3}+1,x^{2},x-1\}$.
\end{questions}
\end{document}
