\documentclass[12pt]{exam}
\usepackage[spanish,mexico]{babel}
\usepackage[latin1]{inputenc}
\usepackage[T1]{fontenc}
%\usepackage{mymacros}
\usepackage{fullpage}
\usepackage{amsmath,amsfonts}
\usepackage{enumerate}
\pagestyle{head}
\extraheadheight{.8in}
\pointname{ puntos}
\lhead{Algebra Lineal, LIFTA, UAEH.\\
Primer Preexamen,\\
22 de agosto de 2013}
\usepackage{fourier}

\DeclareMathOperator{\Ker}{ker}
\newcommand{\RR}{\mathbb{R}}
\newcommand{\setof}[2]{\left\{\,#1\mid #2\,\right\}}

\begin{document}
\hbox to \textwidth{NOMBRE:\enspace\hrulefill}
\bigskip{}

Recuerda que debes explicar tus respuestas. Respuestas con poca o nula
argumentaci�n adecuada recibir�n poco o nada de cr�dito. Escoge 5
preguntas, marcando de manera clara las preguntas
seleccionadas. Tienes 50 minutos para resolverlo.

\begin{questions}
  \question Menciona tres ejemplos de campos (no es necesario
  demostrar que lo son).

  \question Determina si el conjunto
  \begin{equation*}
    \setof{(x,y)\in\RR^{2}}{x^{2}+y=0}
  \end{equation*}
  es un subespacio de $\RR^2$.

  \question Determina si el conjunto
  \begin{equation*}
    \setof{(x,y,z)\in\RR^{3}}{2x+3y-z=x+4y+z}
  \end{equation*}
  es un subespacio de $\RR^3$.

  \question Sea $V=\setof{(x,y)}{x,y\in\RR}$. Define la suma en
  $V$ como $(x,y)+(x',y')=(x+x',y+y')$, y el producto por escalar
  $\lambda\in\RR$ como:
  \begin{equation*}
    \lambda(x,y)=
    \begin{cases}
      (0,0) & \text{si $\lambda=0$}\\
      (\lambda x,\frac{y}{\lambda}) & \text{si $\lambda\ne 0$}
    \end{cases}
  \end{equation*}
  �Es $V$ junto con estas operaciones un espacio vectorial?

  \question Sea $V$ un espacio vectorial sobre un campo $F$. Demuestra
  que si $W_{1}\leq V$ y $W_{2}\leq V$, entonces $W_{1}\cap W_{2}\leq V$.

  \question �Verdadero o falso? Existen espacios vectoriales donde hay
  dos neutros aditivos distintos.

\end{questions}
\end{document}
