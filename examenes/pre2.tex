\documentclass[12pt]{exam}
\usepackage[spanish,mexico]{babel}
\usepackage[latin1]{inputenc}
\usepackage[T1]{fontenc}
%\usepackage{mymacros}
\usepackage{fullpage}
\usepackage{amsmath,amsfonts}
\usepackage{enumerate}
\pagestyle{head}
\extraheadheight{.8in}
\pointname{ puntos}
\lhead{Algebra Lineal, LIFTA, UAEH.\\
Segundo Preexamen,\\
19 de septiembre de 2012}
\usepackage{fourier}

\DeclareMathOperator{\Ker}{ker}
\newcommand{\RR}{\mathbb{R}}
\newcommand{\setof}[2]{\left\{\,#1\mid #2\,\right\}}

\begin{document}
\hbox to \textwidth{NOMBRE:\enspace\hrulefill}
\bigskip{}

Recuerda que debes explicar tus respuestas. Respuestas con poca o nula
argumentaci�n adecuada recibir�n poco o nada de cr�dito. Escoge 5
preguntas, marcando de manera clara las preguntas
seleccionadas. Tienes 50 minutos para resolverlo.

\begin{questions}
  \question �Verdadero o falso? Sean $V$ un espacio vectorial sobre el
  campo $F$ y $S=\{v_{1},v_{2},\ldots,v_{n}\}\subseteq V$ un conjunto
  linealmente dependiente tal que existen
  $\lambda_{1},\lambda_{2},\ldots,\lambda_{n}\in F$ tales que
  \begin{equation*}
    \lambda_{1}v_{1}+\lambda_{2}v_{2}+\cdots+\lambda_{n}v_{n}=0
  \end{equation*}
  Entonces, se deduce que $\lambda_{i}\ne 0$ para alg�n
  $i\in\{1,2,\ldots,n\}$.

  \question Determina si el conjunto
  \begin{equation*}
    \setof{(x,y,z)\in\RR^{3}}{x+y+z=0}
  \end{equation*}
  es un subespacio de $\RR^3$.

  \question Determina si el conjunto
  $\{(1,2,3),(3,4,5),(5,6,7)\}\subseteq\RR^{3}$ es linealmente
  dependiente o linealmente independiente.

  \question Determina si el conjunto
  $\{(1,1,1),(1,0,1),(0,1,0)\}\subseteq\RR^{3}$ es linealmente
  dependiente o linealmente independiente.

 \question Determina si el conjunto
  $\{(1,2,4),(1,3,9),(1,4,16)\}\subseteq\RR^{3}$ es linealmente
  dependiente o linealmente independiente.

 \question Determina si el conjunto
  $\{(0,1,1,1),(1,0,1,1),(1,1,0,1),(1,1,1,0)\}\subseteq\RR^{4}$ es linealmente
  dependiente o linealmente independiente.
\end{questions}
\end{document}
